\chapter{JUSTIFICATIVA}

\par Atualmente devido ao grande volume de dados que, é gerado nas redes,
sugiu a necessidade do gerenciamento de rede, que possa ser eficaz no
gerenciamento de desempenho e na agerencia de configuração. O trafego de dados
nas rede é cada dia mais crescente, com aplicações, portais web e acessos mesmo
dentro da rede interna, fazedo-se assim o gerenciamento de rede o único meio de
gerir as transferências de pacotes em uma rede.Contudo, o projeto sugere uma
aplicação que faça estes gerenciamentos, através de interface amigaveis e
simples, possibilitando a analise e o gerenciamento através da rede.
Este projeto gera também, uma nova fonte de pesquisa bibliográfica sobre
gerenciamento de redes utilizando o SNMP \footnote{Simple Network Management
Protocol \cite{snmp_essencial}}, por ser pioneiro\footnote{Aquele que abre caminho através de
região mal conhecida. \cite{dic_aurelio}} neste tipo de abordagem.
