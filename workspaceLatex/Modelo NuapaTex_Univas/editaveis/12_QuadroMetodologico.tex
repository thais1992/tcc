
\chapter{QUADRO METODOLÓGICO}

\par Conteúdo do quadro metodológico. Perceba a forma que se coloca uma palavra entre aspas: o \LaTeX~oferece muita ``facilitade de formatação''.

Exemplo de código Java:

\begin{lstlisting} [style=custom_Java,caption={[Métodos da classe \texttt{FilmeBean}]{Métodos da classe \texttt{FilmeBean}. \textbf{Fonte:} Elaborado pelos autores.}}, label=fig:metodosclassebean] 	
	public FilmeBean(){  
       //...
   	}	
   	
	public void saveMovie(){
		setListActorSelected();		
		if(this.movieDAO.saveMovieGraph(this.movieTo)){
			FacesContext.getCurrentInstance().addMessage(null, 
			   new FacesMessage("Filme cadastrado com sucesso!")); 
		}else{
			//...
		}		
		this.limpaCampos();
	}
\end{lstlisting}

\par Agora será mostrado o exemplo do uso de fluxo de eventos apresentado no Quadro~\ref{quad:fluxo_evento_cadastro_filme}.

\begin{quadro}[h!]
  \begin{fluxoDeEventos}
  \addTitle{Cadastrar filme}
  \addrow{Ator principal}{Administrador}
  %\addrow{Ator secundário}{Sistema de cartão}
  \addrow{Pré-requisitos}{Estar logado no sistema}

  \startBasicFlow{Ator} {Sistema}
  \addItemOne{Seleciona menu cadastro}
  \addItemOne{Clica na opção cadastrar filme}
  \addItemTwo{Abre interface de cadastro de filme}
  \addItemOne{Preenche formulário}
  \addItemOne{Clica no botão salvar}
  \addItemTwo{Salva e informa sucesso no cadastro}

  \startAlternativeFlow{Fluxo alternativo 1}
  \addItemOne{No item 5, formulário não preenchido}
  \addItemTwo{Exibe mensagem de necessidade de preenchimento de formulário}

  \startAlternativeFlow{Fluxo alternativo 2}
  \addItemOne{No item 6, inserido filme já cadastrado}
  \addItemTwo{Informa mensagem de filme já cadastrado}
\end{fluxoDeEventos}

  \caption[Fluxo de eventos para cadastro de filme]
           {Fluxo de eventos para cadastro de filme. \textbf{Fonte:} Elaborado pelos autores}
  \label{quad:fluxo_evento_cadastro_filme}
\end{quadro}

\par Outro exemplo é ilustrado na Figura~\ref{fig:bluesky}. Neste caso um código XML foi embutido dentro de um ambiente de figura, para que este código seja incluído no índice de figuras adequadamente.
 
\begin{figure}[ht!]
  \begin{lstlisting} [style=custom_XML]
	...
	<context-param>
		<param-name>primefaces.THEME<\param-name>
		<param-value>bluesky<\param-value>
	<\context-param>
	...
  \end{lstlisting}
  \caption[Incluindo o tema \textit{BlueSky} ao contexto do projeto]
          {Incluíndo o tema \textit{BlueSky} ao contexto do projeto. \textbf{Fonte:} Elaborado pelos autores.}
  \label{fig:bluesky}
\end{figure}
