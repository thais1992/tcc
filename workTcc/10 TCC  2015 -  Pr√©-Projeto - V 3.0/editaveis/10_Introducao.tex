\chapter*{INTRODUÇÃO}
\addcontentsline{toc}{chapter}{INTRODUÇÃO}
\stepcounter{chapter} 

\par Segundo \citeonline{Redes_Tanenbaum}, a tecnologia da
Informação começou a evoluir a partir do século XX (Vinte), que se mostrou mais
aberta ao desenvolvimento na coleta, tratamento e distribuição da informação. 
Neste período também nota-se o avanço das tecnologias que foram empregadas
no setor de comunicação.

\par No século anterior teve início a nova era da tecnologia da informação, onde
foi criados métodos de comunicação, a rede de computadores, que surgiu com a
ARPANET na década de 60, uma pequena rede de computadores que seria base da
atual rede mundial de computadores.

\par Para \citeonline{Redes_Forouzan}, A rede de computadores são baseadas
na ideia de comunicação dos dados entre pontos de rede, em que os pacotes são
enviados de um local para o outro. Para o entedimento completo, deve-se ter
conhecimento sobre as arquiteturas de rede, bem como o funcionamento, as
transfências e tipos de dados.

\par Para fazer a transferência de dados na rede de computadores é necessário a
utilização do protocolos, que o SNMP (\textit{Simple Network Management
Protocol}) utiliza como base em pilhas para a transfêrencia de dados.

%\par Para \citeonline{Redes_Forouzan}, o conceito primordial da rede de dados é
% que esta possui a finalidade de transferir dados de um local para outro, denominando-se comunicação de rede. Ao
%observar este conceito faz-se necessário a compreenção da composição física que
% possui uma rede, assim faz ciência do meios e modelos de dados, satisfazendo o conhecimento prévio para
%gerir ou implementar tráfego de dados.

\par Segundo \citeonline{SNMP_Schmidt_Mauro}, torna-se muito complexo
gerenciar uma rede de computadores na atualidade, bem como manter toda rede em
plena execução, dentro dos limites aceitáveis. Neste contexto é que o 
\textit{Simple Network Management Protocol} (SNMP), começa a atuar,
desde sua criação em 1988, com propósito de fazer o gerenciamento da rede de
computadores através do IP (\textit{Internet Protocol}), podendo assim realizar
o gerenciamento de maneira simples, com operações não complexas, facilitando os acessos remotos.
Partindo deste principio, faz-se necessário a criação de meios que faciletem a gerência da rede,
 onde pode ser citado como exemplo, a quantidade e o tráfego de dados gerados nas
empresas. Este projeto tem o intuito de apresentar uma maneira simples de gerenciamento de
redes.

%\par A metodologia do projeto será implementada com ICONIX. Para
% \citeonline{UML_Silva_Videira}, o ICONIX, é um processo para o desenvolvimento de software de maneira prática. 
