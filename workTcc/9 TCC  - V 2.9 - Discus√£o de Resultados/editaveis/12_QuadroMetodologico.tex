
\chapter{QUADRO METODOLÓGICO}


\par Abordamos neste capítulo o método de pesquisa que foi utilizado para a
implementação deste projeto de pesquisa possibilitando sua conclusão.

\par Para \citeonline{Pesquisa_Santos_Candeloro},

\begin{citacao}
	A Expressão “método” remonta à Grécia antiga; methodos (methà + odon),
	siginifica “o caminho para se chegar a um fim”, legando-nos o emprego que
	hoje fazemos, o sentido de eleger um caminho a ser percorrido para se atingir
	um fim. Em se tratando de pesquisa cientifíca, são várias as possibilidades
	para executar uma invetigação e o fim almejado é o da comprovação ou refutação
	das hipóteses levantadas. Em primeiro lugar, deve-se considerar a área na qual
	o acadêmico está inserido e o tema com o qual pretende trabalhar. Após, devese
	construir um Referencial Teórico suficiente que permita o enuciado de um
	problema de pesquisa e suas respectivas variáveis.(p.69). 
\end{citacao}

\par Acompanhando o autor citado anteriormente, podemos concluir que a metodologia de
pesquisa bibliografica fornecera base para o desenvolvimento do projeto, apoiado no problema
de pesquisa e na resolução dos mesmos.

\section{Tipo de Pesquisa}

\par Na implementação do projeto, faz-se necessário a pesquisa que é o meio de 
gerar inovação e conhecimento, e através dela buscar respostas, para 
\citeonline{Pesquisa_Ciribelli}, A pesquisa torna-se fator chave, para a
veracidade e qualidade de um trabalho. Tornando-se assim essencial para o ganho
de conhecimento, bem como é de suma importância na instituição educacional.
  
\par Segundo \citeonline{Pesquisa_Padua}, a pesquisa pode ser definida como, a
meneira vastosa, como um meio de resolver questões não solvidas, exercendo averiguações e questionamentos através
destes meios, podemos assim formar o conhecimento, tornando assim possível o entendimento
da existência que é utilizada com base para ação.

\par Com base neste conceito, anteriormente citados, entende-se que a pesquisa é uma forma
de resolver problemas de pesquisa e gerar conhecimento, embasado neste princípio, a metodologia
aborda, tem como finalidade, apoiar meios de aperfeiçomanentos
do processo de elaboração do projeto .
%
%	*  Tipos de Pesquisa: ( Pesquisa Aplicada / Pesquisa de intervenção )
%	* 
%		* Definir a pesquisa
%		* O porquê da pesquisa
%		* de 3 a 4 paragráfos


\par A metodologia de pesquisa utilizada neste trabalho foi aplicada, onde o
principal foco deste tipo de pesquisa é a geração de conhecimento, dentro de um
espaço curto de tempo, propondo a aplicação prática da metodologia, tendo como
foco um meio de resolver um problema ou uma necessidade de maneira única, cujo o
problema tenha realação e participação com os ambientes locais e regionais.

%falta o tipo de pesquisa -ok 
\par Esta pesquisa utiliza como foco de pesquisa a pesquisa qualitativa, que
segundo \citeonline{Pesquisa_Demo} `` A pesquisa qualitativa busca apenas
realçar partes formalizando com jeito.''




% Contexto da pesquisa

\section{Contexto}

\par Para auxiliar à pesquisa forão utilizados livros, manuais, tutoriais,
trabalhos e artigos acadêmicos e páginas da internet. Utilizando tanto quanto 
para o embasamento teórico tanto quanto para o prático.

\par Com base nas pesquisas elaboradas, foi possível identificar que devido ao
grande volume de dados trafegados na rede, podemos assim complementa que se faz necessário gerir a rede.

\par Segundo \citeonline{Java_Costa_Avancado}, atualmente identificamos o crescimento das redes, bem como
suas estruturas que possui infraestrura complexa. O aumento consideravél de acessos a rede
mundial de computadores á um incitamento impulsionando a criação de novos meios de manter
operantes as rede dentro das condições esperadas.

\par Para \citeonline{Redes_Mosharraf_Forouza},

\begin{citacao}
	Podemos definir o gerenciamento de redes como a tarefa de testar, monitorar,
	configurar e resolver problemas dos componentes de rede com o obejtivo
	de atender um conjunto de requisitos definidos por uma organização. Esses
	requisitos incluem a operação regular e eficiente de rede, proporcionando a
	qualidade de serviços predefinida para os usuários. Para realizar essa tarefa,
	um sistema de gerenciamento de rede usa hardware, software e seres
	humanos.(p.693).
\end{citacao}

\par Conforme \citeonline{Java_Costa_Avancado}, A adminstração de rede é fundamentada na coleta direta de
informação, que constitui toda a estrutura. Com base nesta informações podemos análisar uma
ou mais rede de forma dinâmica, fazendo-se compreender através de seu estado atual.

% detalhes do local de desenvolvimento e de possível aplicação

\par O desenvolvimento da aplicação foi utilizado o eclipse na versão 4.4,
disponível para download no site www.eclipse.org, que é uma ferramenta de
desenvolvimento considerado com IDE (\textit{Integrated Development
Environment}).

%  Participantes : ( Lugar, o que faz, como vai ajudar, pra que serve )
\section{Participantes}

\par Este projeto é escrito por:
\par Israel José da Cunha
\par Graduando em Sistemas de Informação na UNIVAS - Universidade do Vale do
Sapucaí 
\par Professor de informática básica, Intermediária, avançada e aplicada na
Tecnologia.com - Escola Profissionalizante
\par Professor de Tecnologia da Informação no Estado de Minas Gerais - Escola
Estadual Professora Maria Vitorino de Souza.


\par Sob a arientação de:
\par Márcio Emílio Cruz Vono de Azevedo
\par Graduação em Engenharia Elétrica, modalidade Eletrônica pelo Inatel -
Instituto Nacional de Telecomunicações
\par Mestrado em Ciência e Tecnologia da Computação pela UNIFEI - Universidade
Federal de Itajubá
\par Professor no curso de Sistemas de Informação da UNIVÁS - Universidade do
Vale do Sapucaí
\par Professor no curso de Engenharia da Computação do Inatel - Instituto
Nacional de Telecomunicações
\par Especialista em Sistemas no Inatel - Instituto Nacional de Telecomunicações


\section{Instrumentos}

%	*  Instrumentos
%		* Reuniões
%			* Temática - Discussão do tema - Mais aberto permitindo o posicionamento das
%			% integrante * Entrevistas
%			* Estruturadas - Faz a pergunta ela dá a resposta e não mudança de assunto
%			% forçando isso.
%			* Semi Estruturadas -  Entre a estruturada e a aberta o meio termo
%			* Abertas - Totalmente Aberta

%		* Questionários 
%			* Fechados - Respostas em "x" 
%			* Abertos 
%			* Mistos

%Normalmente aplica-se o questionário depois as entrevistas e finalizando com a
% reunião para fechar as opiniões.

%Precisa aparecer no projeto:

%	* Qual o instrumento ?
%	* Qual a definição deste instrumento ? Ex: segundo lakato entrevista é ...
%	* Qual o objetivo do instrumento ? (O porquê do instrumento ?)
%	* Qual a periodicidade ?

\par A pesquisa é um meio de fomentar a coleta dedas, de acordo com
\citeonline{Pesquisa_Lakatos}, os instrumentos de pesquisa contém “desde os
tópicos da entrevista, passando pelo questionário e formulário, 
até os teste ou escala de medidas de opiniões e atitudes”.

% falta
% Instrumentos  -> O que foi usado para obter informações para o desenvolvimento ( Reuniões, entrevistas, questionários, editais, Etc. )
%	1. Definir o instrumento 
%	2. qual o objetivo deste instrumento
%	3. quantos
%	4. Cotucos no trabalho
%		1. Anexos ( Pego e Não modifico ) 
%		2. Apêndice ( Crio ou pego algo e modifico )

\par 


\subsection{Reunião}

\par Serão realizadas reuniões presenciais e virtuais com o professor
orientador, para a discussão dos requisitos, da usabilidade, 
dos processos de engenharia de software e desenvolvimento, questões 
teóricas sobre a aplicabilidade e a orientação para o desenvolvimento do
projeto, será também feito reunião com pessoas que atuam na área para coletar
experiência de campo.

\par Todas as reuniões terão cárter aberto, para livre habitrio dos
participantes.


\par Toda á pesquisa do projeto serão fundamentados em livros, artigos
científicos e acervos digitais.

\par Para \citeonline{Projetos_Cabanis-brewin},
\begin{citacao}
	Documentos incluem planos, registros administrativos, dados técnicos, documentos
	de engenharia e construção, procedimentos, documentos sobre o sistema,
	relatórios e correspondências. Esta seção do plano de gerenciamento do
	projeto identifica os documentos que serão preparados no projeto e estabelece
	a abordagem administrativa, sistemas e procedimentos a serem utilizados para
	gerenciar essa documentação.(p.62)
\end{citacao}

\par A documentação é parte do trabalho, que formula os meios usados no decorrer do projeto,
facilitando a geração de relatórios e emplementação dos requisitos.


\subsection{Metodologia de Desenvolvimento}

\par Será utilizado o modelo de processo chamado ICONIX, por ser um modelo de desnvolvimento
de software iterativo e incremental.

\par De acordo com \citeonline{UML_Silva_Videira}, O ICONIX trata-se de um método de desenvolvimento
de Software, que é divido em quatro pequenos grupo de trabalho, que possuem tempo
de execução. As divisões são: Análise de requisitos, análise e desenho preliminar, desenpenho
e implementação.

\par Ainda segundo \citeonline{UML_Silva_Videira}, Esta metodologia compoem-se em
produzir uma gama de produtos, que descreve com exatidão na perspectiva do sistema, utilizado de meios
incrementais e paralelos, apresentando o ponto de vista Dinâmico e Estático.


\section{Procedimentos}
%   Procedimentos 
%	* Passo-a-passo para o desenvolvimento do software.
%		* Imaginar tudo que vai ter que fazer para fazer o sistema
%			* Todas as ações que poderão ser desenvolvidas
%			* Em tópicos
%			* Em ordem cronológica

% 			FALTA
%Procedimento
%	1. Descrever as ações realizadas para o desenvolvimento
%	2. Descritivo e Detalhado 
%	3. Passo a passo o que foi feito

\begin{itemize}
  \item Análise de Requisitos - Elaboração do plano de execução do projeto de
  desenvolvimento de software.
  \item Implementação do ICONIX - Elaboração dos Digramas de Sistema
  	
  	\begin{itemize}
 	 	\item Análise de requisitos
  		\item Análise e desempenho preliminar
  		\item Desenhos
  		\item Implementação
	\end{itemize}
  	
  \item Revisão do ICONIX
  \item Codificação do projeto 
  \item Teste de codficação 
\end{itemize}

\par O desenvolvimento do aplicativo foi baseado na tecnologia JAVA, através da
IDE Eclipse, o aplicativo tem como finalidade o gerenciamento de dados na rede
de computadores, este tem usabilidade de apresentar graficos e meios de
verificar os resultados de transferências de dados e aplicações que estejam em
execução no instante de seu respectivo uso.

\par Foi elaborado as fases do ICONIX, que são as seguintes, Análise de
requisitos, Análise e desempenho preliminar e os diagramas.

\par Após a criação dos digramas deu-se início ao desenvolvimento da aplicação,
através da Linguagem de programação JAVA.


%\section{Resultados}

%Resultados
%	1. O que foi obtido a partir das ações ( Descritivo e detalhado )

%Poder fazer isso -> Procedimentos e Resultado
%	1. Exceto tipo de pesquisa as outras seções deverão estar no tempo verbal
	% "PASSADO" 2. Texto Técnico porém científico
% Pode usar em formato de tabela.

%\par Os resoltados demonstram que para cada ação feita existe um resultado
%equivalente. Na tabela á seguir estão exemplificados todas as ações feitas.

%\begin{small}
%\begin{tabular}{clr}
%\hline
% Procedimento & Resultados  \\
%\hline
% 	1-  & 1-  \\
% 	2- & 2-  \\
% 	3-  & 3-   \\
% 	4-   & 4-  \\ 
% 	5-  & 5-  \\ 
%\hline
%\end{tabular}


%\end{small}





% Cronograma

\section{Cronograma}

\begin{table}[!htpb]
\centering

% definindo o tamanho da fonte para small
% outros possíveis tamanhos: footnotesize, scriptsize
\begin{small} 
  
% redefinindo o espaçamento das colunas
\setlength{\tabcolsep}{8pt} 

% \cline é semelhante ao \hline, porém é possível indicar as colunas que terão essa a linha horizontal
% \multicolumn{10}{c|}{Meses} indica que dez colunas serão mescladas e a palavra Meses estará centralizada dentro delas.

\begin{tabular}{|l|c|c|c|c|c|c|c|c|c|}\hline

\textbf{Ações / Meses} & Jun & Jul & Ago & Set & Out & Nov & Dez
\\
\hline

Qualificação & X &  &  &  &  &  &  \\ \hline

Redação do quadro teórico & X & X &  &  &  &  &  \\ \hline

Redação do quadro metodológico & X & X & X &  &  &  &  \\ \hline

Desenvolvimento do sistema & X & X & X & X & X & X & X \\ \hline

Redação da discussão dos resultados & X & X & X & X & X & X & X \\ \hline

Pré-banca &  &  &  &  &  & X &  \\ \hline

Redação da introdução e considerações finais &  &  & X & X &  &  & \\\hline

Formatação final &  &  &  &  &  & X & X \\ \hline

Banca de defesa &  &  &  &  &  &  & X \\ \hline

Correções finais &  &  &  &  &  &  & X \\ \hline

Entrega de Capa Dura &  &  &  &  &  &  & X \\ \hline

Reuniões & X & X & X & X & X & X & X \\ \hline

Revisões Bibliograficas & X & X & X & X & X & X & X \\ \hline

Revisão Textual & X & X & X & X & X & X & X \\ \hline


\end{tabular} 
\end{small}
\caption{Cronograma das atividades previstas}
\label{t_cronograma}
\end{table} 


\section{Orçamento}

% Orçamento
%	* Listar todas as despesas com o TCC
%		* Folhas 
%		* Tintas
%		* Encadernação
%		* Internet
%		* Transporte
%		* Capa Dura
%		* Estipular um valor médio ( não ficar muito baixo e nem muito auto )



\begin{small}
\begin{tabular}{clr}
\hline
 Despesas & Valor Previsto (R\$)   \\
\hline
 	Impressão & R\$ 150,00  \\
 	Encadernação & R\$ 70,00  \\
 	Impressão Capa Dura  & R\$ 200,00  \\
 	Livros  & R\$ 475,00  \\ 
 	Ebook  & R\$ 254,00  \\ \hline
 	\textbf{Total} & \textbf{R\$ 1.149,00} \\
\hline
\end{tabular}


\end{small}


% Descritivo detalhado

%\par
















