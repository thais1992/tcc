\chapter{OBJETIVO}

\par Este projeto tem como objetivos os tópicos seguintes.

\section{Objetivo Geral}

\par Desenvolver uma aplicação que seja de fácil iteração, atuando como ferramenta de suporte
para o administrador de rede, com foco em gerência de configuração e desempenho, com
modelos de análise gráfica, baseados em MRTG (\textit{The Multi Router Trafic
Grafic}). O monitoramento dos equipamentos de rede será obtido pelo Nagios,
fazendo o uso do SNMP (\textit{Simple Network Management Protocol}, para
facilitar o acesso das informações.

\section{objetivo Específico}

\par Através do protocolo SNMP, analisar o tráfego de dados internos da
\textit{Local Area Network} (LAN).

\par Utilizar o protocolo IP para obeter configurações dos terminais.

\par Tranferir os dados necessários utilizando o TCP,

\par Criação do gráfico de consumo local de pacotes, através do MRTG.

\par Facilitar a análise de desempenho da rede.

\par Análise de dados, com o tempo de respostas e o tráfego na rede, podendo, a
partir desta análise, definir o horário de maior fluxo de transferência de pacotes e qual setor requer maior
atenção.

\par Identificar o IP que está consumindo maior largura de banda e seus
respectivos links.

\par Gerar relatórios periódicos e gráficos de períodos específico na
contabilidade de dados.

\par Identificar os computadores na rede.

\par Detectar os sistemas operacionais, seus serviços e programas instalados e
em execução.

\par Gerir com mais eficácia as aplicações presentes nos sistemas operacionais.
Identificar quais arquivos e as respectivas datas que estes foram abertos no computador.