\chapter{METODOLOGIA}

\par Abordamos neste capítulo o método de pesquisa que será utilizado para a implementação
deste projeto de pesquisa possibilitando sua conclusão.

\par Para \citeonline{Pesquisa_Santos_Candeloro},

\begin{citacao}
	A Expressão “método” remonta à Grécia antiga; methodos (methà + odon),
	siginifica “o caminho para se chegar a um fim”, legando-nos o emprego que
	hoje fazemos, o sentido de eleger um caminho a ser percorrido para se atingir
	um fim. Em se tratando de pesquisa cientifíca, são várias as possibilidades
	para executar uma invetigação e o fim almejado é o da comprovação ou refutação
	das hipóteses levantadas. Em primeiro lugar, deve-se considerar a área na qual
	o acadêmico está inserido e o tema com o qual pretende trabalhar. Após, devese
	construir um Referencial Teórico suficiente que permita o enuciado de um
	problema de pesquisa e suas respectivas variáveis.(p.69). 
\end{citacao}

\par Acompanhando o autor citado anteriormente, podemos concluir que a metodologia de
pesquisa bibliografica fornecera base para o desenvolvimento do projeto, apoiado no problema
de pesquisa e na resolução dos mesmos.

\section{Tipo de Pesquisa}

\par Na implementação do projeto, faz-se necessário a pesquisa que é o meio de 
gerar inovação e conhecimento, e através dela buscar respostas, para 
\citeonline{Pesquisa_Ciribelli}, A pesquisa torna-se fator chave, para a
veracidade e qualidade de um trabalho. Tornando-se assim essencial para o ganho
de conhecimento, bem como é de suma importância na instituição educacional.
  
\par Segundo \citeonline{Pesquisa_Padua}, a pesquisa pode ser definida como, a
meneira vastosa, como um meio de resolver questões não solvidas, exercendo averiguações e questionamentos através
destes meios, podemos assim formar o conhecimento, tornando assim possível o entendimento
da existência que é utilizada com base para ação.

\par Com base neste conceito, anteriormente citados, entende-se que a pesquisa é uma forma
de resolver problemas de pesquisa e gerar conhecimento, embasado neste princípio, a metodologia
aborda, pesquisa bibliográfica, tem como finalidade, apoiar meios de aperfeiçomanentos
do processo de elaboração do projeto .

\section{Contexto}

\par Devido ao grande volume de dados trafegados na rede, podemos assim complementa
que se faz necessário gerir a rede.

\par Segundo \citeonline{Java_Costa_Avancado}, atualmente identificamos o crescimento das redes, bem como
suas estruturas que possui infraestrura complexa. O aumento consideravél de acessos a rede
mundial de computadores á um incitamento impulsionando a criação de novos meios de manter
operantes as rede dentro das condições esperadas.

\par Para \citeonline{Redes_Mosharraf_Forouza},

\begin{citacao}
	Podemos definir o gerenciamento de redes como a tarefa de testar, monitorar,
	configurar e resolver problemas dos componentes de rede com o obejtivo
	de atender um conjunto de requisitos definidos por uma organização. Esses
	requisitos incluem a operação regular e eficiente de rede, proporcionando a
	qualidade de serviços predefinida para os usuários. Para realizar essa tarefa,
	um sistema de gerenciamento de rede usa hardware,software e seres humanos.(p.693).
\end{citacao}

\par Conforme \citeonline{Java_Costa_Avancado}, A adminstração de rede é fundamentada na coleta direta de
informação, que constitui toda a estrutura. Com base nesta informações podemos análisar uma
ou mais rede de forma dinâmica, fazendo-se compreender através de seu estado atual.


\section{Resgistros Documentais}

\par Toda á pesquisa do projeto serão fundamentados em livros, artigos
científicos e acervos digitais.

\par Para \citeonline{Projetos_Cabanis-brewin},
\begin{citacao}
	Documentos incluem planos, registros administrativos, dados técnicos, documentos
	de engenharia e construção, procedimentos, documentos sobre o sistema,
	relatórios e correspondências. Esta seção do plano de gerenciamento do
	projeto identifica os documentos que serão preparados no projeto e estabelece
	a abordagem administrativa, sistemas e procedimentos a serem utilizados para
	gerenciar essa documentação.(p.62)
\end{citacao}

\par A documentação é parte do trabalho, que formula os meios usados no decorrer do projeto,
facilitando a geração de relatórios e emplementação dos requisitos.


\subsection{Metodologia de Desenvolvimento}

\par Será utilizado o modelo de processo chamado ICONIX, por ser um modelo de desnvolvimento
de software iterativo e incremental.

\par De acordo com \citeonline{UML_Silva_Videira}, O ICONIX trata-se de um método de desenvolvimento
de Software, que é divido em quatro pequenos grupo de trabalho, que possuem tempo
de execução. As divisões são: Análise de requisitos, análise e desenho preliminar, desenpenho
e implementação.

\par Ainda segundo \citeonline{UML_Silva_Videira}, Esta metodologia compoem-se em
produzir uma gama de produtos, que descreve com exatidão na perspectiva do sistema, utilizado de meios
incrementais e paralelos, apresentando o ponto de vista Dinâmico e Estático.












