%%This is a very basic article template.
%%There is just one section and two subsections.
\documentclass[12pt,a4paper]{abntex2}
\renewcommand{\rmdefault}{ptm}
\usepackage[T1]{fontenc}
\usepackage[utf8]{inputenc}

\begin{document}

\chapter{Gerência de Redes SMNP}

\par
O sistema tem por base o protocolo de gerência SMNP, combinado com diversas
tecnologias possam ser inseridas ao projeto, como o Nagios, JavaEE. MRTG e JSF,
tambem será utilizado recursos de rede que terão como base o TCP e SMNP e, para
poder todos os benefícios oferecidos pelos meios propostos

\par
O sistema tera que possui um interface amigável e parametrizável para
homologação dos dados, assim todas as redes serão atendidas com enfase nos
seguintes processos:


\section{Gerência de Configuração}
\par
Identificação de computadores na rede, assim como quais os tipos de equipamento
estão ligados a estes computadores, detectar os sistemas operacionais assim como seus serviços e programas em execução e
instalados, podendo assim gerenciar melhor as aplicaçães presentes nos sistemas
operacionais, podendo identificar quais arquivos e as datas que foram abetos no computador.

\par
\section{Gerência de Segurança}

Apresentar o mapa da rede, identificando os pontos de redes e apresentando em
modo gráfico todos os itens compostos na estrutura, assim como seus IPs, MAC, 
e Nome do computador na rede, com suas respetivas configuraçães (Mascarra, IP,
Gatway), podendo também fazer buscas através do IP trazendo os dados da máquina.


\section{Gerência de Falha}
\par
Verifição dos erros ocorridos durante o tempo de uso da rede, como cabos com
problemas na transferência de dados, falhas na conexção com a internet e
comunicação com os servidores, podendo assim gerar relatórios com tempo de espera e logs de erros do sistema, 
disparar lembretes, no caso de falhas que possam interferir no uso, como internet, falhas e servidores.

\section{Gerência de Desempenho}
\par
Serão analisados dados com tempo de respostas e trafego na rede, podendo definir
assim qual o horário de maior tráfego na rede e qual parte da rede precisa de
maior atenção. Identifica também qual o IP que está consumindo maior largura de
banda e seus respectivos links de download e upload, gerando relatórios
periódicos e gráficos de um período específico.

\end{document}
