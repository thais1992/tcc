\chapter*{INTRODUÇÃO}
\addcontentsline{toc}{chapter}{INTRODUÇÃO}
\stepcounter{chapter} 

%*******************************************************************************************************
% TEMA

\par No ponto de vista de \citeonline{Redes_Tanenbaum}, a tecnologia da
informação começou a evoluir a partir do século XX, que se mostrou mais 
aberta ao desenvolvimento na coleta, tratamento e distribuição da 
informação. Neste período também nota-se o avanço das tecnologias 
que foram empregadas no setor de comunicação de computadores.

\par No século anterior teve início a nova era da tecnologia da informação, onde
foi criados métodos de comunicação, a rede de computadores, que surgiu com a
ARPANET na década de 60, uma pequena rede de computadores que seria base da
atual rede mundial de computadores.

\par Para \citeonline{Redes_Forouzan}, A rede de computadores são baseadas
na ideia de comunicação dos dados entre pontos de rede, em que os pacotes são
enviados de um local para o outro. Para o entedimento completo, deve-se ter
conhecimento sobre as arquiteturas de rede, bem como o funcionamento, as
transfências e tipos de dados.

\par Para fazer a transferência de dados na rede de computadores é necessário a
utilização do protocolos, que o SNMP utiliza como base de pilhas e para a
transfêrencia de dados.

%-----------------------------------------------------------------------------------------------------
%Versão 0.5
% a finalidade primordial da 
%rede de dados é que esta possui a finalidade de transferir dados de 
%um local para outro, denominando-se comunicação de rede. Ao observar 
%este conceito faz-se necessário a compreenção da composição física 
%que possui uma rede, assim faz ciência do meios e modelos de dados, 
%satisfazendo o conhecimento prévio para gerir ou implementar 
%tráfego de dados.
%-----------------------------------------------------------------------------------------------------
\par De acordo com \citeonline{Artigo_SNMP_Lecia}, O SNMP (\textit{Simple
Network Management Protocol}), foi implementado no padrão IETF 
(\textit{Internet Engineering Task Force }), na década de 80, que possuia apenas
o modelo de gerenciamento e monitoração de IP, este modelo e classificado como
SGMP (\textit{Simple Gateway Management Protocol}). 

\par Para \citeonline{Artigo_RFC_1157}, o  SNMP é dividido conforme sua
evolução. Desde sua primeira versão, em 1989, que foi batizado como SNMPv1. Em
1992, foi incorporado a RMON (\textit{Remote Monitoring}) a primeira versão do
SNMP,já em 1993, foi então lançada a versão 2 do SNMP,com novos recursos de
segurança, performace, comunicação entre gerentes e confiabilidade. Em 1996, foi
incorporado a SNMPv2 o \textit{Community Security}, 
e a criação da MIB (\textit{Management Information Base}) e a atualização da
RMON (\textit{Remote Monitoring}) para a segunda versão, batizado como RNOMv2,
criando assim a versão SNMPv2c. Em 1998, foi melhorada a versão do SNMPv2,
através do \textit{User Security Model}, gerando a versão 3 do SNMP, adiconando novos recursos de segunraça como
autenticação, privacidade e controle de acesso.

\par Segundo \citeonline{SNMP_Schmidt_Mauro}, torna-se muito complexo
gerenciar uma rede de computador na atualidade, bem como manter toda 
rede em plena execução, dentro dos limites aceitáveis. 
Neste contexto é que o \textit{Simple Network Management Protocol} (SNMP) 
começa a atuar, desde sua criação em 1988, com propósito de fazer o 
gerenciamento da rede de computadores através do IP  
(\textit{Internet Protocol}), podendo assim realizar
o gerenciamento de maneira simples, com operações não complexas, 
facilitando os acessos remotos. Partindo deste principio, faz-se 
necessário a criação de meios que faciletem a gerência da rede, onde pode ser
citado como exemplo, a quantidade e o tráfego de dados gerados nas empresas. 
Este projeto tem o intuito de apresentar 
uma maneira simples de gerenciamento de redes.

%*******************************************************************************************************
% Delimitação

\par Este projeto vai auxiliar no gerenciamento de redes, onde
grande parte dos profissionais empregam grande parte do tempo para obter resultados mínimos com o
desenvolvimento da aplicação, que visa diminir o tempo e os custos gerados na adiministração.


%*******************************************************************************************************
% Trabalhos na área

\par Para \citeonline{Artigo_SNMP_Lecia}, O SNMP é um meio de auxílio no
gerenciamento de rede de computadores, que faz uso do protocolo UDP para a
transferência de dados, possibilitando ao admistrador a
localização e a correção de falhas ou problemas dentro do ambiente de rede, O
SNMP possuí agentes (MIB) que facilitam a analise de dados. 
A MIB, é o conjunto de objetos que são geridos, este objetos são variáveis que
possuem uma estrutura hieráquica (em formato de árvore), onde encontramos nas
folhas (da arvore) a identificação da estrutura e definição da informação, onde
a definição e a estrutura é determinada pela linguagem ASN.1 (\textit{Abstract Syntax Notation One}),
regulamentada e criada pela ITU (\textit{International Telecommunication
Union}). A MIB faz uso de uma estrutura de gerenciamento de informação o SMI
(\textit{Structure of Management Information}), que determina quais e como uma
informação gerencial será agrupada e denominada, que determina a extensão e os
tipos de dados que serão utilizados na sintaxe da MIB.

%*******************************************************************************************************
% Objetivos ( Geral e específico )


%\section{Objetivo Geral}

\par Este projeto tem como objetivo, desenvolver uma aplicação de gerência de
configuração e desempenho de rede, utilizando o protocolo SNMP (\textit{Simple Network Management Protocol}).

%\section{objetivo Específico}
\par Delimitando com o obejtivos específicos o gerenciamento através do
protocolo SNMP, analisar o tráfego de dados internos da \textit{Local Area Network} (LAN); 

\par Utilizar o protocolo IP para obeter configurações dos terminais.

\par Identificar o IP que está consumindo maior largura de banda e seus
respectivos links.

\par Identificar os computadores na rede.

%-----------------------------------------------------------------------------------------------------
%\par Tranferir os dados necessários utilizando o TCP,
%\par Criação do gráfico de consumo local de pacotes, através do MRTG.
%\par Facilitar a análise de desempenho da rede.
%\par Análise de dados, com o tempo de respostas e o tráfego na rede, podendo, a
%partir desta análise, definir o horário de maior fluxo de transferência de
% pacotes e qual setor requer maior atenção.
%\par Gerar relatórios periódicos e gráficos de períodos específico na
%contabilidade de dados.
%\par Detectar os sistemas operacionais, seus serviços e programas instalados e
%em execução.
%\par Gerir com mais eficácia as aplicações presentes nos sistemas operacionais.
%Identificar quais arquivos e as respectivas datas que estes foram abertos no
% computador.
%-----------------------------------------------------------------------------------------------------

%*******************************************************************************************************
% Justificativa
\par A justificativa para a realização deste trabalho é embasada em que,
atualmente, os acesso às informações geram grande tráfego na rede, 
através de acesso locais e dentro da rede mundial de computadores, criando a necessidade de gerir as transferências
de dados pela rede de maneira eficaz e eficiente, com ênfase no desempenho e na
configuração. A ferramenta de gerência de configuração e desenpenho de redes de
computadores, possibilita o conjunto de novas fontes bibliográficas para
pesquisas, com pioneirismo no desenvolvimento de aplicações específicas para a administrar a 
rede de computadores, os equipamentos e os pacotes e datagramas
\footnote{Datagramas é a reunião de partes únicas de 
trantransferência de dados dentro de uma rede de computadores.}


%-----------------------------------------------------------------------------------------------------

%\par Atualmente devido ao grande volume de dados que, é gerado nas redes,
%sugiu a necessidade do gerenciamento de rede, que possa ser eficaz no
%gerenciamento de desempenho e na agerencia de configuração. O trafego de dados
%nas rede é cada dia mais crescente, com aplicações, portais web e acessos mesmo
%dentro da rede interna, fazedo-se assim o gerenciamento de rede o único meio de
%gerir as transferências de pacotes em uma rede.Contudo, o projeto sugere uma
%aplicação que faça estes gerenciamentos, através de interface amigaveis e
%simples, possibilitando a analise e o gerenciamento através da rede.
%Este projeto gera também, uma nova fonte de pesquisa bibliográfica sobre
%gerenciamento de redes utilizando o SNMP \footnote{Simple Network Management
%Protocol \cite{snmp_essencial}}, por ser pioneiro\footnote{Aquele que abre
% caminho através de região mal conhecida. \cite{dic_aurelio}} neste tipo de abordagem.
 
%-----------------------------------------------------------------------------------------------------
