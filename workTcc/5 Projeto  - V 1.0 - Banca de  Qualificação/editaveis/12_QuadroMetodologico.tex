
\chapter{QUADRO METODOLÓGICO}

%
%	*  Tipos de Pesquisa: ( Pesquisa Aplicada / Pesquisa de intervenção )
%	* 
%		* Definir a pesquisa
%		* O porquê da pesquisa
%		* de 3 a 4 paragráfos


\par A metodologia de pesquisa utilizada neste trabalho será aplicada, onde o
principal foco deste tipo de pesquisa é a geração de conhecimento, dentro de um
espaço curto de tempo, propondo a aplicação prática da metodologia, tendo como
foco um meio de resolver um problema ou uma necessidade de maneira única, cujo o
problema tenha realação e participação com os ambientes locais e regionais.

%\par Segundo \citeonline{Pesquisa_Dulce_Perdigão}, 

%\begin{citacao}
		
%\end{citacao}




% Contexto da pesquisa

\section{Contexto}


\par Para auxiliar à pesquisa serão utilizados livros, manuais, tutoriais,
trabalhos e artigos acadêmicos e páginas da internet. Utilizando tanto quanto 
para o embasamento teórico tanto quanto para o prático.


%  Participantes : ( Lugar, o que faz, como vai ajudar, pra que serve )
\section{Participantes}

\par Este projeto é escrito por:
\par Israel José da Cunha
\par Graduando em Sistemas de Informação na UNIVAS - Universidade do Vale do
Sapucaí 
\par Professor de informática básica, Intermediária, avançada e aplicada na
Tecnologia.com - Escola Profissionalizante
\par Professor de Tecnologia da Informação no Estado de Minas Gerais - Escola
Estadual Professora Maria Vitorino de Souza.


\par Sob a arientação de:
\par Márcio Emílio Cruz Vono de Azevedo
\par Graduação em Engenharia Elétrica, modalidade Eletrônica pelo Inatel -
Instituto Nacional de Telecomunicações
\par Mestrado em Ciência e Tecnologia da Computação pela UNIFEI - Universidade
Federal de Itajubá
\par Professor no curso de Sistemas de Informação da UNIVÁS - Universidade do
Vale do Sapucaí
\par Professor no curso de Engenharia da Computação do Inatel - Instituto
Nacional de Telecomunicações
\par Especialista em Sistemas no Inatel - Instituto Nacional de Telecomunicações


%	*  Instrumentos
%		* Reuniões
%			* Temática - Discussão do tema - Mais aberto permitindo o posicionamento das
%			% integrante * Entrevistas
%			* Estruturadas - Faz a pergunta ela dá a resposta e não mudança de assunto
%			% forçando isso.
%			* Semi Estruturadas -  Entre a estruturada e a aberta o meio termo
%			* Abertas - Totalmente Aberta

%		* Questionários 
%			* Fechados - Respostas em "x" 
%			* Abertos 
%			* Mistos

%Normalmente aplica-se o questionário depois as entrevistas e finalizando com a
% reunião para fechar as opiniões.

%Precisa aparecer no projeto:

%	* Qual o instrumento ?
%	* Qual a definição deste instrumento ? Ex: segundo lakato entrevista é ...
%	* Qual o objetivo do instrumento ? (O porquê do instrumento ?)
%	* Qual a periodicidade ?

\section{Instrumentos}

\par A pesquisa é um meio de fomentar a coleta dedas, de acordo com
\citeonline{Pesquisa_Lakatos}, os instrumentos de pesquisa contém “desde os
tópicos da entrevista, passando pelo questionário e formulário, 
até os teste ou escala de medidas de opiniões e atitudes”.

\subsection{Reunião}

\par Serão realizadas reuniões presenciais e virtuais com o professor
orientador, para a discussão dos requisitos, da usabilidade, 
dos processos de engenharia de software e desenvolvimento, questões 
teóricas sobre a aplicabilidade e a orientação para o desenvolvimento do
projeto, será também feito reunião com pessoas que atuam na área para coletar
experiência de campo.

\par Todas as reuniões terão cárter aberto, para livre habitrio dos
participantes.


\section{Procedimentos}
%   Procedimentos 
%	* Passo-a-passo para o desenvolvimento do software.
%		* Imaginar tudo que vai ter que fazer para fazer o sistema
%			* Todas as ações que poderão ser desenvolvidas
%			* Em tópicos
%			* Em ordem cronológica



\begin{itemize}
  \item Análise de Requisitos - Elaboração do plano de execução do projeto de
  desenvolvimento de software.
  \item Implementação do ICONIX - Elaboração dos Digramas de Sistema
  	
  	\begin{itemize}
 	 	\item Análise de requisitos
  		\item Análise e desempenho preliminar
  		\item Desenhos
  		\item Implementação
	\end{itemize}
  	
  \item Revisão do ICONIX
  \item Codificação do projeto 
  \item Teste de codficação 
\end{itemize}


% Cronograma

\section{Cronograma}

\begin{table}[!htpb]
\centering

% definindo o tamanho da fonte para small
% outros possíveis tamanhos: footnotesize, scriptsize
\begin{small} 
  
% redefinindo o espaçamento das colunas
\setlength{\tabcolsep}{8pt} 

% \cline é semelhante ao \hline, porém é possível indicar as colunas que terão essa a linha horizontal
% \multicolumn{10}{c|}{Meses} indica que dez colunas serão mescladas e a palavra Meses estará centralizada dentro delas.

\begin{tabular}{|l|c|c|c|c|c|c|c|c|c|}\hline

\textbf{Ações / Meses} & Jun & Jul & Ago & Set & Out & Nov & Dez
\\
\hline

Qualificação & X &  &  &  &  &  &  \\ \hline

Redação do quadro teórico & X & X &  &  &  &  &  \\ \hline

Redação do quadro metodológico & X & X & X &  &  &  &  \\ \hline

Desenvolvimento do sistema & X & X & X & X & X & X & X \\ \hline

Redação da discussão dos resultados & X & X & X & X & X & X & X \\ \hline

Pré-banca &  &  &  &  &  & X &  \\ \hline

Redação da introdução e considerações finais &  &  & X & X &  &  & \\\hline

Formatação final &  &  &  &  &  & X & X \\ \hline

Banca de defesa &  &  &  &  &  &  & X \\ \hline

Correções finais &  &  &  &  &  &  & X \\ \hline

Entrega de Capa Dura &  &  &  &  &  &  & X \\ \hline

Reuniões & X & X & X & X & X & X & X \\ \hline

Revisões Bibliograficas & X & X & X & X & X & X & X \\ \hline

Revisão Textual & X & X & X & X & X & X & X \\ \hline


\end{tabular} 
\end{small}
\caption{Cronograma das atividades previstas}
\label{t_cronograma}
\end{table} 


\section{Orçamento}

% Orçamento
%	* Listar todas as despesas com o TCC
%		* Folhas 
%		* Tintas
%		* Encadernação
%		* Internet
%		* Transporte
%		* Capa Dura
%		* Estipular um valor médio ( não ficar muito baixo e nem muito auto )



\begin{small}
\begin{tabular}{clr}
\hline
 Despesas & Valor Previsto (R\$)   \\
\hline
 	Impressão & R\$ 150,00  \\
 	Encadernação & R\$ 70,00  \\
 	Impressão Capa Dura  & R\$ 200,00  \\
 	Livros  & R\$ 475,00  \\ 
 	Ebook  & R\$ 254,00  \\ \hline
 	\textbf{Total} & \textbf{R\$ 1.149,00} \\
\hline
\end{tabular}


\end{small}


% Descritivo detalhado

%\par



