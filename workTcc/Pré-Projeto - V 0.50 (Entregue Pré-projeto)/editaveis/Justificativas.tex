\chapter{JUSTIFICATIVA}

\par Atualmente, os acesso às informações geram grande tráfego na rede, através de acesso
locais e dentro da rede mundial de computadores, criando a necessidade de gerir as transferências
de dados pela rede de maneira eficaz e eficiente, com ênfase no desempenho e na configuração.

\par A ferramenta possibilita o conjunto de novas fontes bibliográficas para pesquisas, com
pioneirismo no desenvolvimento de aplicações específicas para a administrar a rede de computadores,
os equipamentos e os datagramas \footnote{Datagramas (ou trama ou pacote) é a reunião de partes únicas 
de trantransferência de dados dentro de uma rede de computadores.}

\par Este projeto tem o intuito de suprir a dificuldade no gerenciamento de redes, onde grande
parte dos profissionais empregam grande parte do tempo para obter resultados mínimos com o
desenvolvimento da aplicação, que visa diminir o tempo e os custos gerados na adiministração.

%\par Atualmente devido ao grande volume de dados que, é gerado nas redes,
%sugiu a necessidade do gerenciamento de rede, que possa ser eficaz no
%gerenciamento de desempenho e na agerencia de configuração. O trafego de dados
%nas rede é cada dia mais crescente, com aplicações, portais web e acessos mesmo
%dentro da rede interna, fazedo-se assim o gerenciamento de rede o único meio de
%gerir as transferências de pacotes em uma rede.Contudo, o projeto sugere uma
%aplicação que faça estes gerenciamentos, através de interface amigaveis e
%simples, possibilitando a analise e o gerenciamento através da rede.
%Este projeto gera também, uma nova fonte de pesquisa bibliográfica sobre
%gerenciamento de redes utilizando o SNMP \footnote{Simple Network Management
%Protocol \cite{snmp_essencial}}, por ser pioneiro\footnote{Aquele que abre
% caminho através de região mal conhecida. \cite{dic_aurelio}} neste tipo de abordagem.


