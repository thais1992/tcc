% --- resumo em português ---

\begin{OnehalfSpacing} 

\noindent \imprimirAutorCitacaoMaiuscula. {\bfseries\imprimirtitulo}. {\imprimirdata}.  Monografia -- Curso de {\MakeUppercase\imprimircurso}, {\imprimirinstituicao}, {\imprimirlocal}, {\imprimirdata}.

\vspace{\onelineskip}
\vspace{\onelineskip}
\vspace{\onelineskip}
\vspace{\onelineskip}

\begin{resumo}
~\\
%início do texto do resumo
\noindent Este trabalho apresenta um meio de gerência de redes, será um sistema
que terá como base para tecnologias como o Nágios, MRTG e JSF, também será
utilizado recursos de rede com base no TCP e SNMP, para poder manipular todos os
benefícios que estas tecnologias podem eferecer.
Terá a identificação de maquinas na rede, assim como suas aplicações que
estejam sendo usadas no momento, bem como os sites que estejam sendo acessados,
será abordado também o mapeamento físico da rede, com quantos computadores estão
ligados a rede, quais as suas configurações e o quanto eles estão utilizando da
rede, bem como o horário onde a rede possui mais trafego.
Na rede será gerenciado, os pontos de conexão de cada máquina e equipamento que
esteja conectado dentro da rede LAN assim os itens da rede serão apresentados
de forma gráfica.
\ldots

%fim do texto do resumo
\vspace{\onelineskip}
\vspace*{\fill}
\noindent \textbf{Palavras-chave}: \imprimirPalavraChaveUm. \imprimirPalavraChaveDois. \imprimirPalavraChaveTres.
\vspace{\onelineskip}
\end{resumo}

\end{OnehalfSpacing}
