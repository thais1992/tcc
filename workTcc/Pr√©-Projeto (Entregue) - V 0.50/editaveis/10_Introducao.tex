\chapter{INTRODUÇÃO}

\par No ponto de vista de \citeonline{Redes_Tanenbaum}, a tecnologia começou a evoluir a
partir do século XX, que se mostrou mais aberta ao desenvolvimento na coleta, tratamento e distribuição
da informação. Neste período também nota-se o avanço das tecnologias que foram empregadas
no setor de comunicação.

\par Para \citeonline{Redes_Forouzan}, o conceito primordial da rede de dados é que esta
possui a finalidade de transferir dados de um local para outro, denominando-se comunicação de rede. Ao
observar este conceito faz-se necessário a compreenção da composição física que possui uma
rede, assim faz ciência do meios e modelos de dados, satisfazendo o conhecimento prévio para
gerir ou implementar tráfego de dados.

\par Segundo \citeonline{SNMP_Schmidt_Mauro}, torna-se muito complexo
gerenciar uma rede na atualidade, bem como manter toda rede em plena execução, dentro dos limites aceitáveis. Neste
contexto é que o \textit{Simlpe Network Management Protocol} começa a atuar
desde sua criação em 1988, com propósito de fazer o gerenciamento da rede de
computadores através do IP (\textit{Internet Protocol}), podendo assim realizar
o gerenciamento de maneira simples, com operações não complexas, facilitando os acessos remotos.
Partindo deste principio, faz-se necessário a criação de meios que faciletem a gerência da rede,
 onde pode ser citado como exemplo, a quantidade e o tráfego de dados gerados nas
empresas. Este projeto tem o intuito de apresentar uma maneira simples de gerenciamento de
redes.

\par A metodologia do projeto será implementada com ICONIX. Para \citeonline{UML_Silva_Videira},
 o ICONIX, é um processo para o desenvolvimento de software de maneira prática. Esta metodologia
é iquivalente ao XP (\textit{Extreme Programming}) e ao RUP (\textit{Rational Unified
Process}), que baseia-se na simplicidade de seu desenvolvimento.
